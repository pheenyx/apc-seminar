\documentclass[11pt]{beamer}

%% Zeichenkodierung

\usepackage[latin1]{inputenc}

%% Deutsches Sprachpaket
\usepackage[german]{babel}

%% Grafik-Einbindung
\usepackage{graphicx}

%% Mathe-Einbindung
\usepackage{amsmath,amsfonts,amssymb,bbm}

%% Schriftart �ndern auf Computer Modern Sans Serif
\usepackage[T1]{fontenc}
\renewcommand*\familydefault{\sfdefault}

%% Beamer Template
\usetheme{Luebeck}

%% Navigationssymbole ausblenden
\setbeamertemplate{navigation symbols}{}

%% Seitenzahl und Gesamtanzahl in Fuߟleiste
\setbeamertemplate{footline}[frame number]

%% Metadaten des Dokuments
\title[RMQ]{Range Minimum Query}
\author{Michael Mardaus \and Felix Lauenroth}
\date{21. Mai 2013}

%% Einzug nach Absatz und Zwischenraum
\setlength{\parindent}{0pt}
\setlength{\parskip}{8pt}

\begin{document}

\thispagestyle{empty}

\begin{frame}
  \titlepage
\end{frame}


\begin{frame}{Inhaltsverzeichnis}
\tableofcontents[pausesections]
\end{frame}

\section{Allgemeines}

\subsection{�berblick}

\begin{frame}{�berblick}
  \begin{definition}[Range Minimum Query]
    
    \(   RMQ_{A}(l,r) = arg \)
    
    \begin{array}[t]{ll} min \\
      l \leq k \leq r 
    \end{array}
    \( A[k] \)
  \pause  
  
  Finden des kleinsten Elements in einem Interval!   
  \end{definition}
\end{frame}

\subsection{Naiver Ansatz}

\begin{frame}
\begin{block}{Lorem Ipsum}
  Lorem Ipsum
\end{block}
\end{frame}


\section{Effiziente Algorithmen}

\subsection{Speicher optimal}

\begin{frame}
\begin{block}{Lorem Ipsum}
Lorem Ipsum
\end{block}
\end{frame}


\subsection{Zugriff optimal}

\begin{frame}

\end{frame}

\section{Unser Problem}

\subsection{Skyline}

\begin{frame}

\end{frame}

\subsection{Naiver Ansatz}

\begin{frame}

\end{frame}

\subsection{optimierte Version}

\begin{frame}

\end{frame}


\thispagestyle{empty}


\end{document}
