\documentclass[11pt]{beamer}

%% Zeichenkodierung

\usepackage[latin1]{inputenc}

%% Deutsches Sprachpaket
\usepackage[german]{babel}

%% Grafik-Einbindung
\usepackage{graphicx}

%% Mathe-Einbindung
\usepackage{amsmath,amsfonts,amssymb,bbm}

%% Schriftart �ndern auf Computer Modern Sans Serif
\usepackage[T1]{fontenc}
\renewcommand*\familydefault{\sfdefault}

%% Beamer Template
\usetheme{Luebeck}

%% Navigationssymbole ausblenden
\setbeamertemplate{navigation symbols}{}

%% Seitenzahl und Gesamtanzahl in Fuߟleiste
\setbeamertemplate{footline}[frame number]

%% Metadaten des Dokuments
\title[RMQ]{Range Minimum Query}
\author{Michael Mardaus \and Felix Lauenroth}
\date{21. Mai 2013}

%% Einzug nach Absatz und Zwischenraum
\setlength{\parindent}{0pt}
\setlength{\parskip}{8pt}

%% Michaels Addons
\newcommand*{\Oh}{\mathcal{O}}

\begin{document}

\thispagestyle{empty}

\begin{frame}
  \titlepage
\end{frame}


\begin{frame}{Inhaltsverzeichnis}
\tableofcontents[pausesections]
\end{frame}

\section{Allgemeines}

\subsection{�berblick}

\begin{frame}{�berblick}
  \begin{definition}[Range Minimum Query]
    
    
    \[\text{RMQ}_{A}(\ell,r) = \text{arg} \ \min_{\ell \leq k \leq r } A[k] \]
    
  \pause
  \only<2->{
  In Worten:\\
  Gesucht ist der Index $k$ des kleinsten Elements $A[k]$ in einem gegebenen Bereich ($\ell\leq k\leq r$) innerhalb eines Arrays $A$.}
  \end{definition}
  Bild Beispiel
\end{frame}

\subsection{Naiver Ansatz}

\begin{frame}
\begin{block}{Idee 1 - trivial}
 Durch das gegebene Array (von $\ell$ bis $r$) durch iterieren und dabei das Minimum merken $\Rightarrow$ Querytime: $\Oh(n)$
\end{block}
 \pause
 \begin{block}{Idee 2 - mit Preprocessing}
 Preprocessing -- also das Aufbereiten eines zus�tzlichen Arrays, f�r mehrere zuk�nftige Queries -- kann zu sublinearen Zugriffszeiten f�hren.\\
 Erste Idee: Abspeichern der Minima jeder Kombination von Intervallgrenzen in einer 2-dim. Matrix (� la Adjazenzmatrix)\\
 $\Rightarrow$ Preprocessingtime: $\Oh(n^3)$, Querytime: $\Oh(1)$
 \end{block}

 
\end{frame}

\begin{frame}[fragile]
\begin{block}{Idee 3 - mit dynamischer Programmierung}
�hnliche Idee, allerdings mit dynamischer Programmierung im Preprocessing Schritt.\\
  $\Rightarrow$ Preprocessingtime: $\Oh(n^2)$, Querytime $\Oh(1)$.
\end{block}
\vspace{-6mm}
  \pause
  \begin{verbatim}
 void trivial3(int M[N][N], int A[N], int N){
      for (i = 0; i < N; i++)
          M[i][i] = i;
      for (i = 0; i < N; i++)
          for (j = i + 1; j < N; j++)
              if (A[M[i][j - 1]] < A[j])
                  M[i][j] = M[i][j - 1];
              else
                  M[i][j] = j;
  }
  \end{verbatim}
  \vspace{-5mm}
\end{frame}


\section{Effiziente Algorithmen}
\frame{\sectionpage}

\subsection{Speicheroptimiert}

\begin{frame}[fragile]
\begin{block}{$\sqrt n$ Teile Algo}

Algorithmus der das Array in $\sqrt n$ Teile teilt und von diesen Teilen jeweils das Minimum speichert.\\[5mm]
Bild\\[5mm]
$\Rightarrow$ Preprocessingtime: $\Oh(n)$, Querytime: $\Oh(\sqrt n)$
\end{block}
\end{frame}


\subsection{Zugriffsoptimiert}

\begin{frame}[fragile]
\begin{block}{Algo mit Sparse Table}
Algorithmus erstellt eine $n \times \log n$ Matrix in der Teilarrays der L�nge $2^k$ per dynamischer Programmierung berechnet werden\\[5mm]
Bild von Array\\[5mm]

Code des dynmischen Programmierungschritts:
\vspace{-5mm}
\begin{verbatim}
if (A[M[i][j - 1]] < A[M[i + (1 << (j - 1))][j - 1]])
    M[i][j] = M[i][j - 1];
else
    M[i][j] = M[i + (1 << (j - 1))][j - 1];
\end{verbatim}
$\Rightarrow$ Preprocessingtime: $\Oh(n\cdot\log n)$, Querytime: $\Oh(1)$
\end{block}
\end{frame}

\section{Problemset}

\subsection{Beschreibung}

\begin{frame}
\begin{block}{Skyline}
Gesucht wird die gr��tm�gliche, zusammenh�ngende, rechteckige Fl�che innerhalb einer gegebenen Skyline.\\
Die Skyline besteht aus H�usern der Breite 1 und gegebener H�he (Input pro Haus).
\end{block}

erkl�rendes Bild \\[5mm]
\end{frame}

\subsection{L�sung}
\begin{frame}
Beschreibung:\\
\begin{enumerate}
\item Suchen des Minimums im aktuellen Intervall.
\item Berechnen der Fl�che unterhalb dieses Minimums\\ (Minimum * Intervallbreite) und vergleichen mit bisher gefundener maximalen Fl�che.
\item Teilen des Intervalls in rechtes und linkes Teilproblem (von diesem Minimum aus gesehen).
\item In diesen beiden Teilproblemen wieder Schritt 1 anwenden.
\end{enumerate}
\end{frame}

\begin{frame}[fragile]
\begin{block}{Pseudo-Code}
\vspace{-3mm}
\begin{verbatim}
preprocessing();         // compute all minima
\end{verbatim}
\pause
\vspace{-7mm}
\begin{verbatim}
area = 0; max = 0;       // initalize
Q = { Interval(0, N) };
\end{verbatim}
\pause
\vspace{-7mm}
\begin{verbatim}
while ( !Q.empty() ){
	Interval(left, right) = Q.pop();
    min = RMQ( Interval(left, right) );
    area = min * (right - left + 1);
    if ( area > max ) max = area;
    Q.add(Intervall(left, min - 1 ));
    Q.add(Intervall(min + 1, right ));
}
\end{verbatim}
\end{block}
\end{frame}

\begin{frame}
ggf. rekursiver Algorithmus von Skyline?
\end{frame}

\thispagestyle{empty}


\end{document}
