\documentclass[11pt]{beamer}

%% Zeichenkodierung

\usepackage[latin1]{inputenc}

%% Deutsches Sprachpaket
\usepackage[german]{babel}

%% Grafik-Einbindung
\usepackage{graphicx}

%% Mathe-Einbindung
\usepackage{amsmath,amsfonts,amssymb,bbm}

%% Schriftart �ndern auf Computer Modern Sans Serif
\usepackage[T1]{fontenc}
\renewcommand*\familydefault{\sfdefault}

%% Beamer Template
\usetheme{Luebeck}

%% Navigationssymbole ausblenden
\setbeamertemplate{navigation symbols}{}

%% Seitenzahl und Gesamtanzahl in Fuߟleiste
\setbeamertemplate{footline}[frame number]

%% Metadaten des Dokuments
\title[RMQ]{Range Minimum Query}
\author{Michael Mardaus \and Felix Lauenroth}
\date{21. Mai 2013}

%% Einzug nach Absatz und Zwischenraum
\setlength{\parindent}{0pt}
\setlength{\parskip}{8pt}

\begin{document}

\thispagestyle{empty}

\begin{frame}
  \titlepage
\end{frame}


\begin{frame}{Inhaltsverzeichnis}
\tableofcontents[pausesections]
\end{frame}

\section{Allgemeines}

\subsection{Overlays}

\begin{frame}{Effekthascherei}
\alt<7->{Gutgemeinter Tipp:}{Tipp:}
\begin{center}
  \alert<8>{Nicht} \pause zu \pause viele \pause \textbf{\Large Effekte} \pause verwenden\pause!
\end{center}
\end{frame}

\subsection{Mehrere Spalten}

\begin{frame}{Zweispaltig}
\begin{columns}
\column{.50\textwidth}
1. Spalte
\column{.50\textwidth}
\begin{enumerate}
\item 2. Spalte
\item 2. Spalte
\end{enumerate}
\end{columns}
\end{frame}

\begin{frame}{Dreispaltig}
\begin{columns}
\column{.33\textwidth}
1. Spalte
\column{.33\textwidth}
2. Spalte
\column{.33\textwidth}
3. Spalte
\end{columns}
\end{frame}

\section{Strukturierung}

\subsection{Listen}

\begin{frame}{Ungeordnete Listen}
\begin{itemize}
  \item Ein Punkt
  \item Noch ein Punkt
  \item \ldots
\end{itemize}
\end{frame}

\begin{frame}{Geordnete Listen}
\begin{enumerate}
  \item Ein Punkt
  \item Noch ein Punkt
  \item \ldots
\end{enumerate}
\end{frame}

\subsection{Beschreibungen}

\begin{frame}{Beschreibungen}
\begin{description}
  \item[Gruppe A] Ein Punkt
  \item[Gruppe B] Noch ein Punkt
  \item[Gruppe C] \ldots
\end{description}
\end{frame}

\subsection{Bl�cke}

\begin{frame}{einfacher Block}
\begin{block}{Blocktitel}
  Block
\end{block}
\end{frame}

\begin{frame}{Alert-Block}
\begin{alertblock}{Blocktitel}
  Alert-Block
\end{alertblock}
\end{frame}

\begin{frame}{Example-Block}
\begin{exampleblock}{Blocktitel}
  Example-Block
\end{exampleblock}
\end{frame}

\subsection{Mathematisches}

\begin{frame}{Definition}
\begin{definition}[von Etwas]
  Definition
\end{definition}
\end{frame}

\begin{frame}{Lemma}
\begin{lemma}[�ber Etwas]
  Lemma
\end{lemma}
\end{frame}

\begin{frame}{Theorem}
\begin{theorem}[�ber Etwas]
  Theorem
\end{theorem}
\end{frame}

\begin{frame}{Beweis}
\begin{proof}[von Etwas]
  \texttt{AUTOPROOF = TRUE}
\end{proof}
\end{frame}

\thispagestyle{empty}

\begin{frame}
\begin{center}
\Huge \textbf{Viel Spa� und Erfolg!}
\end{center}
\end{frame}

\end{document}
